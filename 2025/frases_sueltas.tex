Preclinical research utilizing animal models is a crucial part of biomedical science, serving to investigate and validate new biomarkers, imaging agents, and therapeutic interventions prior to clinical trial initiation. Today,


Preclinical molecular imaging offers several
benefits, notably the ability to explore both ana- tomical and molecular aspects of biological pro- cesses. This is made possible through hybrid imaging
technologies that merge functional imaging, such as positron emission tomography (PET)
or single-photon emission computed
tomography (SPECT), with structural imaging techniques
like magnetic resonance imaging
(MRI) or computed tomography (CT).


In PAI, light absorption spectra of tissue components, both endogenous chromophores such as
oxyheamoglobin and deoxyheamoglobin, lipids, and water, and exogenous agents such as small molecule dyes (e.g. indocyanine green (ICG) and methylene blue), nanoparticles, or genetic reporter agents (Goldman 1990, Wang et al 2020)) are exploited.

By facilitating precise investigations in animal models, cellular samples, and tissue specimens, preclinical research not only deepens our understanding of disease mechanisms but also lays the groundwork for advancing diagnostic and therapeutic innovations in biomedical science


The application of PET imaging to translational studies of placental drug transfer and fetal function during pregnancy has been of interest since the mid-1980s.1 Traditional in vivo or ex vivo methods in animal models may provide information on the placental transfer and fetal disposition of drugs; however, they often require invasive chronic catheterization or sampling of amniotic fluid.2 In contrast, minimally invasive PET imaging offers the opportunity for dynamic and longitudinal measures in fetal compartments, potentially providing a more detailed assessment of PK/PD in vivo. Similarly, functional studies in the fetus, such as changes in protein expression, metabolic function, or drug occupancy, often require postmortem tissue, while PET provides quantitative measurements with minimal physiological disturbance to the mother-fetus dyad. Importantly, PET also carries limitations, namely relatively poor spatial (2–4 mm) and temporal (minutes) resolution in comparison to MRI and ultrasound (<1 mm, <1 second, respectively), as well as exposure to ionizing radiation. However, compared to other noninvasive imaging methods used for fetus visualization, PET imaging has the key advantages of high sensitivity and selectivity for specific molecular processes.