Preclinical research utilizing animal models is a crucial part of biomedical science, serving to investigate and validate new biomarkers, imaging agents, and therapeutic interventions prior to clinical trial initiation. Today,


Preclinical molecular imaging offers several
benefits, notably the ability to explore both ana- tomical and molecular aspects of biological pro- cesses. This is made possible through hybrid imaging
technologies that merge functional imaging, such as positron emission tomography (PET)
or single-photon emission computed
tomography (SPECT), with structural imaging techniques
like magnetic resonance imaging
(MRI) or computed tomography (CT).


In PAI, light absorption spectra of tissue components, both endogenous chromophores such as
oxyheamoglobin and deoxyheamoglobin, lipids, and water, and exogenous agents such as small molecule dyes (e.g. indocyanine green (ICG) and methylene blue), nanoparticles, or genetic reporter agents (Goldman 1990, Wang et al 2020)) are exploited.

By facilitating precise investigations in animal models, cellular samples, and tissue specimens, preclinical research not only deepens our understanding of disease mechanisms but also lays the groundwork for advancing diagnostic and therapeutic innovations in biomedical science