\documentclass[10pt,a4paper]{article}
\usepackage[utf8]{inputenc}
\usepackage[T1]{fontenc}
\usepackage{amsmath}
\usepackage{amssymb}
\usepackage{graphicx}
\usepackage{color,soul}
\usepackage{easyReview}
\usepackage{booktabs}
\usepackage{multirow}
\usepackage[section]{placeins}
\usepackage[hidelinks]{hyperref}
\usepackage[backend=biber, style=numeric, citestyle=numeric-comp, sorting=none, hyperref=true, maxnames=20]{biblatex}
\usepackage[nottoc]{tocbibind} % la opcion nottoc hace que el indice no aparezca
% en la tabla de contenidos
\usepackage[a4paper, margin=30mm]{geometry}
\usepackage{authblk}
\usepackage{lineno}
\usepackage{float}
\usepackage{indentfirst}
\usepackage[none]{hyphenat}
\usepackage{acronym}	
\usepackage{nomencl}
\usepackage{lineno}
\usepackage[breakable]{tcolorbox}

\usepackage[spanish]{babel} % para que no diga "chapter", y diga "capítulo"
\usepackage{pdfpages} % Para insertar PDFs
\usepackage{tabularx}

\addto\captionsspanish{\renewcommand{\tablename}{Tabla}}

% \linenumbers

%%% carpeta de figuras %%%%%%
\graphicspath{{figs/}}

\addbibresource{bibliog_2025.bib}

%%%%%%%% algunas cosas para renderizar mejor las ecuaciones %%%%%%%%%%%%
\DeclareMathSizes{10}{9}{7}{5}

% overline con mas espacio entre letras y barra
\newcommand*\oline[1]{%
	\vbox{%
		\hrule height 0.5pt%                  % Line above with certain width
		\kern0.4ex%                          % Distance between line and content
		\hbox{%
			\kern-0.1em%                        % Distance between content and left side of box, negative values for lines shorter than content
			\ifmmode#1\else\ensuremath{#1}\fi%  % The content, typeset in dependence of mode
			\kern-0.1em%                        % Distance between content and left side of box, negative values for lines shorter than content
		}% end of hbox
	}% end of vbox
}

\newcommand{\norma}[1]{\left\|#1\right\|}
\newcommand{\abector}[2]{\oline{#1#2}}
\newcommand{\pdv}[2]{\dfrac{\partial #1}{\partial #2}}

%\linenumbers

\linespread{1.5} % espacio de interlineado
% Set the space after paragraphs
\setlength{\parskip}{1em} % Adjust the length as needed


\title{Momentum \\ Tecnología de imagen multi-modal para investigación pre-clínica (TIMPRE) \\ Memoria anual 2025}
\author{Guillermo G. Cosarinsky Markman}
\date{}

%%%%%%%%%%%%%%%%%%%%%%%%%%%%%%%%%%%%%%%%%%%%%%%%%%%%%%%%%%%%%%%%%%%%%%%%%%%%%%%%%%%%%%%%%%%%%

\begin{document}	
\maketitle

\tableofcontents

\section{Introducción}\label{sec:intro}
La presente memoria resume el trabajo realizado durante 2025 en el marco del proyecto TIMPRE, financiado por la convocatoria Momentum CSIC. El objetivo principal del proyecto es mi formación posdoctoral orientada al desarrollo de competencias digitales, a través de la participación activa en el diseño y avance de tecnología multimodal de imagen para investigación preclínica, en colaboración con los grupos participantes. Este entorno multidisciplinar, que integra experiencia en imagen médica entre los grupos de ITEFI y grupos externos, permite potenciar sinergias y aprovechar las oportunidades emergentes en el campo de la investigación biomédica.

El grupo en el que trabajo se especializa en tecnologías de ultrasonido, y es en imagen ultrasónica donde se centra mi contribución principal. Otros investigadores con los que colaboramos aportan experiencia en técnicas complementarias como la tomografía por emisión de positrones (PET) y la imagen fotoacústica.

La investigación preclínica constituye un pilar esencial en el desarrollo de nuevos fármacos y terapias, exigiendo tecnologías de imagen especializadas que superen en resolución y prestaciones a las empleadas en el entorno clínico. En este contexto, la imagen por ultrasonido destaca como una técnica versátil y bien establecida, capaz de proporcionar información anatómica y fisiológica en tiempo real, sin radiación ionizante y con costes relativamente bajos de adquisición y mantenimiento \cite{Moran2020}.

La integración de ultrasonido con otras modalidades de imagen —como PET, resonancia magnética (MRI) y fotoacústica— permite obtener información complementaria sobre morfología, función y metabolismo en los modelos animales \cite{Wu2018, Rovera2025}. La fusión de estas técnicas incrementa la sensibilidad y especificidad de los estudios, abriendo nuevas posibilidades para el seguimiento de procesos biológicos complejos, la caracterización de patologías y la evaluación de terapias innovadoras. Este enfoque multimodal resulta clave para avanzar en la investigación preclínica y favorecer la futura transferencia de tecnologías al ámbito clínico.

Esta memoria se organiza de la siguiente manera: en la sección \ref{sec:estado_arte} se presenta una reseña del estado del arte en tecnologías de imagen para investigación preclínica, poniendo especial énfasis en las técnicas de ultrasonido. La sección \ref{sec:nsi} describe la implementación del método de imagen ultrasónica Null Subtraction Imaging (NSI) \cite{Agarwal2019_NSI}, su aplicación a un conjunto de datos de imagen funcional cerebral de rata proporcionado por Neuro-Electronics Research Flanders (NERF), así como el trabajo preliminar orientado a extender el método a arrays matriciales. En la sección \ref{sec:actividades} se sintetizan los artículos publicados durante 2025, enfocados en el desarrollo de métodos para la obtención de \highlight{imágenes en sistemas con interfaces refractantes}. Finalmente, la sección \ref{sec:conclusiones} expone las principales conclusiones y las perspectivas de trabajo futuro.


\highlight{Durante mi doctorado hice cosas de imagen 3D con array matricial en NDT blablabla
En el este proyecto, imagen preclinica multimodal, ultrasonidos 3D, intersección de las disciplinas, NDT e imagen médica cosas en común blablablabla}

\subsection{Estado del arte}\label{sec:estado_arte}

generalidades...
Ultrafast imaging, power doppler, fucntional imaging
Trasncranial imaging
Artherosclerosis
Cancer?

El desarrollo y la validación de nuevas terapias y biomarcadores en biomedicina dependen en gran medida de la investigación preclínica, para lo cual las tecnologías de imagen juegan un papel fundamental \cite{Rovera2025}. La Tabla \ref{tab:metodos-imagen} resume los principales métodos de imagen biomédica empleados tanto en investigación clínica como preclínica, incluyendo tomografía computarizada (CT), resonancia magnética (MRI), tomografía por emisión de positrones (PET), tomografía por emisión de fotón único (SPECT), ultrasonido (US) y fotoacústica (PA) \cite{Bushberg2021}. Cada una de estas técnicas ofrece ventajas específicas en cuanto a resolución espacial, sensibilidad y tipo de información obtenida (anatómica, funcional o molecular).

\begin{table}[H]
	\small
	\centering
	\caption{Resumen de métodos de imagen biomédica \cite{Bushberg2021}. Los acrónimos corresponden a los nombres en inglés, los cuales son los más comúnmente utilizados en la literatura científica.} 
	\begin{tabularx}{\textwidth}{X|X|X}
		\\
		\hline
		\textbf{Técnica (acrónimo)} & \textbf{Principio físico} & \textbf{Aplicaciones} \\
		\hline\hline
		Tomografía Computarizada (CT) & Rayos X & Anatomía, lesiones óseas, pulmón \\
		\hline
		Resonancia Magnética (MRI) & Campos magnéticos de radiofrecuencia & Cerebro, músculo, vasos sanguíneos, tejido blando \\
		\hline
		Tomografía por Emisión de Positrones (PET) & Radiotrazadores y detección de rayos gamma & Metabolismo, oncología, imagen funcional \\
		\hline
		Tomografía Computarizada por Emisión de Fotón Único (SPECT) & Radioisótopos y detección de rayos gamma & Cardiología, neurología, imagen funcional, perfusión tisular \\
		\hline
		Ultrasonido (US) & Ondas acústicas (ultrasonido) & Vasos sanguíneos, corazón, tejidos blandos, obstetricia \\
		\hline
		Fotoacústica (PA) & Excitación con luz láser y detección de ultrasonido & Oxigenación, angiogénesis, tumor, piel \\
		\hline
	\end{tabularx}
	\label{tab:metodos-imagen}
\end{table}

En el contexto preclínico, la integración de modalidades multimodales —como la combinación de PET con MRI o CT, y ultrasonido con fotoacústica— permite obtener información complementaria sobre la morfología, función y metabolismo de tejidos y órganos \cite{Wu2018, Perez-Liva_2025}. Por ejemplo, la imagen PET aplicada a un modelo animal porcino permite estudiar la actividad metabólica asociada a patologías como la arteroesclerosis, siendo capaz de identificar lesiones vulnerables a partir de la expresión de enzimas glucolíticas en distintos tipos celulares \cite{Nogales2025}. A su vez, las placas ateroscleróticas pueden ser caracterizadas mediante US, lo cual permite evaluar de manera no invasiva la morfología y composición de la pared arterial \cite{Yao2023}. Esto demuestra el potencial de la integración de de distaintas modalidades de imagen. Además, la incorporación de algoritmos de inteligencia artificial (IA) al análisis de las imágenes permite una detección más automatizada y fiable de lesiones vulnerables\cite{Wang2025}. Más allá de este ejemplo sobre arteroesclerosis, la investigación en el uso de herramientas de inteligencia artificial (IA) en imagen preclínica está en pleno auge, como lo demuestra el estudio de \cite{DeRosa2023}

Otro ejemplo destacado de complementariedad entre distintas modalidades de imagen lo constituye el estudio de procesos fisiológicos y patológicos durante el desarrollo fetal. En \cite{Torben2025}, se ha utilizado PET para investigar, en un modelo murino\footnote{El término “modelo murino” hace referencia al uso de ratones o ratas como animales de experimentación en estudios biomédicos.}, la fisiología fetal y la dinámica de intercambio materno-fetal. Por otro lado, en \cite{Tanvir2025}, la tomografía fotoacústica permitió evaluar de manera no invasiva parámetros funcionales y morfológicos del corazón fetal, como la saturación de oxígeno y la variabilidad de la frecuencia cardíaca, ante la exposición prenatal al alcohol.

Un aspecto clave en el avance de la imagen biomédica preclínica es el desarrollo de técnicas capaces de evaluar la función y la microvasculatura tisular. En este contexto, ha surgido el campo de las Imágenes de Ultrasonido de Súper-Resolución (SRUS), cuyo objetivo es superar el límite de difracción clásico del ultrasonido (aproximadamente la mitad de la longitud de onda) para distinguir objetos o estructuras más cercanas \cite{Christensen2020, Couture2018}. Las técnicas de SRUS han logrado mejorar el poder de resolución del ultrasonido hasta en un factor de 10 con respecto a este límite.

Dentro de las SRUS, la Microscopía de Localización por Ultrasonido (ULM) es una técnica precursora y la más común. Se inspira en la microscopía óptica de localización y se basa en la localización precisa de microburbujas individuales inyectadas como agentes de contraste \cite{Couture2018}. Al acumular estas localizaciones a lo largo de un gran número de fotogramas, la ULM puede reconstruir imágenes compuestas de súper-resolución. Esta técnica ha permitido visualizar vasos con diámetros de hasta 8--30~$\mu$m en diversas aplicaciones, incluyendo el cerebro de ratas, riñones, tumores y en cánceres y cerebro humanos. 

En este panorama, la Imagen de Súper-Resolución basada en Eritrocitos (SURE) \cite{Jensen2024_SURE} representa una nueva e innovadora aproximación dentro de las SRUS. A diferencia de la ULM, la SURE utiliza los eritrocitos (glóbulos rojos) circulantes como blancos de imagen, eliminando la necesidad de agentes de contraste externos. 

Paralelamente, métodos como el ultrasonido funcional (fUS) permiten monitorizar en tiempo real cambios en el volumen sanguíneo cerebral como medida indirecta de la actividad neuronal, facilitando estudios de neuroimagen funcional en modelos animales, incluso en condiciones fisiopatológicas como el accidente cerebrovascular \cite{Montaldo2022_fus, Brunner2023}. La fUS se basa en los principios del Doppler de onda pulsada para detectar el movimiento de la sangre, proporcionando una alta resolución espacio-temporal (100~$\mu$m, 100~ms) y un amplio campo de visión. La capacidad de las SRUS, incluyendo SURE, para visualizar la arquitectura de la microvasculatura con una resolución muy superior a la del ultrasonido convencional, posibilita la detección de vasos de pequeño calibre y el análisis detallado de redes vasculares en órganos como el riñón, complementando y potencialmente mejorando la comprensión de los mecanismos hemodinámicos subyacentes a la actividad neuronal que fUS observa. Un ejemplo de integración de fUS y ULM se presenta en \cite{Jones2024} donde la utilización de agentes de contraste, que aumentan la relación señal ruido (SNR), permitieron obtener imagen funcional cerebral a través del cráneo. Un ejemplo de integración entre fUS y ULM se muestra en \cite{Jones2024}, donde el empleo de agentes de contraste, que mejoran la relación señal-ruido (SNR), permitió obtener imágenes funcionales del cerebro a través del cráneo. Este procedimiento representa un desafío, principalmente por la atenuación y la aberración de fase que genera el hueso craneal.

%Un aspecto clave en el avance de la imagen biomédica preclínica es el desarrollo de técnicas capaces de evaluar la función y la microvasculatura tisular. Métodos como el ultrasonido funcional (fUS) permiten monitorizar en tiempo real cambios en el volumen sanguíneo cerebral como medida indirecta de la actividad neuronal, facilitando estudios de neuroimagen funcional en modelos animales, incluso en condiciones fisiopatológicas como el accidente cerebrovascular \cite{Montaldo2022_fus, Brunner2023}. Por otro lado, las técnicas avanzadas de ultrasonido como la imagen de super-resolución basada en eritrocitos (SURE) \cite{Jensen2024_SURE}, permiten visualizar la arquitectura de la microvasculatura con una resolución muy superior a la del ultrasonido convencional y sin necesidad de agentes de contraste, posibilitando la detección de vasos de pequeño calibre y el análisis detallado de redes vasculares en órganos como el riñón. 


\section{Actividades realizadas}\label{sec:actividades}

\subsection{Publicaciones en revistas JCR}\label{sec:publicaciones}
surface AIM
deep echo
oscar sparse? lo dejo para el año proximo 

\subsection{Congresos internacionales}\label{sec:congresos}
forum
ius?

\subsection{Cursos}

\subsection{Desarrollos iniciales con el método de beamforming avanzado Null Subtraction Imaging (NSI)}\label{sec:nsi}
\subsubsection{Patrón lateral en campo lejano 3D}


\section{Conclusiones y trabajo futuro}\label{sec:conclusiones}

\newpage
\appendix

% Apéndice A en página nueva, centrado y grande
%\thispagestyle{empty} % Opcional: sin número de página
\vspace*{\fill}
\begin{center}
	{\LARGE \textbf{Apéndice A}}
\end{center}
\vspace*{\fill}
\newpage

% Insertar el PDF en la siguiente página
%\includepdf[pages=-]{paper_AIM.pdf}

\end{document}